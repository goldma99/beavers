To estimate the impact of beaver reintroduction on agricultural productivity, I obtain data on (1) the Tayside beaver expansion from three comprehensive regional surveys conducted over a decade, (2) parcel-level crop cover\todo{fill in years details if acquired}, (3) agricultural functioning, including \todo{Fill in when I get variables from Ag Stats}, and (4) river level from a network of in-situ hydrometry monitors across Scotland. To describe and quantify the extent of potentially impacted areas, I employ crop cover data, exploiting its spatiotemporal variation to test for crop switching behavior in response to beaver entry. To model the impact of beaver reintroduction on agricultural productivity, I use the beaver expansion surveys and agricultural functioning data. To assess potential damage mechanisms, I use the hydrometry data.  

\subsection{Beaver Expansion}
\begin{itemize}
    \item Describe survey methods, with attention to causal inference concerns
    \item Describe how to translate survey observations to population estimates
\end{itemize}

\subsubsection{Filling in Beaver Expansion Gaps}

\begin{itemize}
    \item Describe approach to addressing gaps in the panel, either by 
\end{itemize}

\subsection{Crop Cover}
\begin{itemize}
    \item Describe source/methods of UKCEH crop cover maps
\end{itemize}

\subsection{Agricultural Activity}

\begin{itemize}
    \item Describe variables, with special attention to interpretation of Standard Output (is SO even viable, since it's an average value?)
\end{itemize}

\subsection{River Levels}
\begin{itemize}
    \item Describe hydrometry network and history
    \item Describe measurements collected, with special attention to the interpretation of river level
\end{itemize}

% \subsection{Remote Sensing of Floods}