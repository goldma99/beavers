In Table \ref{table:beaver_main_ag_share_all_samples}, I report estimates from Equation \ref{eq:main_beaver_eq} of agriculture land use changes following beaver entry. 

In column (1), which includes all treatment cohorts and grid cells, beaver entry increases the share of land devoted to agriculture by 2.8 p.p. (22.4\% relative to baseline). The effect is driven by cropping changes in landscape cells adjacent to rivers (cols (2), (4), and (6)), which supports anecdotal accounts that beaver effects remain localized to within tens of meters of their residence. Across the table, I vary the composition of the treatment group. In columns (1) and (2), I include all treated units. In columns (3) and (4), I remove the cohort treated in 2020. In columns (5) and (6), I further remove the 2017 treatment cohort. Across the cohort samples, the effect remains stable. 

\begin{table}[htb]
\captionlistentry[table]{}
\label{table:beaver_main_ag_share_all_samples}
\centering
Table \ref{table:beaver_main_ag_share_all_samples} \\
Beaver impacts \\
\begin{threeparttable}
\begin{tabulary}{\textwidth}{l*{5}{c}@{}}
\toprule \toprule
\noalign{\smallskip}
\ExpandableInput{\tabPath/beaver_main/beaver_main_DVag_share_Sall_samples_panel.tex}
\noalign{\smallskip}
\midrule \bottomrule
\end{tabulary}
\medskip
\begin{tablenotes}[flushleft]
\setlength\labelsep{0pt}
\item
\footnotesize
\justify
Notes: Estimation results from Equation \eqref{eq:main_beaver_eq}.
Each regression includes grid cell and time period fixed effects.
Samples vary by column.
Standard errors are clustered at the grid cell level.  \\
\mbox{*} 0.10 ** 0.05 *** 0.01
\end{tablenotes}
\end{threeparttable}
\end{table}


To test whether beaver entry effects vary by land type, I divide the sample by three soil types described in \ref{sec:data}. I report the results in Table \ref{table:beaver_sample_soil_Soverall_Cweather_controls}. Consistent with reporting on beavers affecting intensive cropping operations more than grasslands, in columns (3) and (4), the effect appears to be driven by activity in land classified as suitable for arable cropping. My preferred specification is column (4), which includes all treatment cohorts but restricts the sample to only on-river cells with arable cropping soil. Here, beaver entry increases the proportion of cropped land by 4.6 p.p. (11.3\% relative to the baseline, equivalent to 4.5 hectares). Columns (5) and (6), which report results on the subsample with soil suitable only for improved grassland or rough grazing, suggest beavers caused little to no change in agricultural cropping. This is consistent with the fact that such areas are barely cropped (with a mean of less than 1\% land area cropped). A further useful placebo test is shown in columns (7) and (8), which report results in non-agricultural soil areas, including built, water, and unmapped areas. If the remote-sensed land use data trustworthy, one would expect to see no change in cropping in such areas. The results point to such a placebo test succeeding. 

\begin{table}[htb]
\captionlistentry[table]{}
\label{table:beaver_sample_soil_Soverall_Cweather_controls}
\centering
Table \ref{table:beaver_sample_soil_Soverall_Cweather_controls} \\
Beaver impacts \\
\begin{threeparttable}
\begin{tabulary}{\textwidth}{l*{9}{c}@{}}
\toprule \toprule
\noalign{\smallskip}
\ExpandableInput{\tabPath/beaver_sample_soil/beaver_sample_soil_DVag_share_Soverall_Cweather_controls_panel.tex}
\noalign{\smallskip}
\midrule \bottomrule
\end{tabulary}
\medskip
\begin{tablenotes}[flushleft]
\setlength\labelsep{0pt}
\item
\footnotesize
\justify
Notes: Estimation results from Equation \eqref{eq:main_beaver_eq}.
Each regression includes grid cell and time period fixed effects.
Samples vary by column. Regression includes average two-meter temperature and average total precipitation covariates.
Standard errors are clustered at the grid cell level. \\
\mbox{*} 0.10 ** 0.05 *** 0.01
\end{tablenotes}
\end{threeparttable}
\end{table}


In case a small number of outlier units are driving the large positive results in Tables \ref{table:beaver_main_ag_share_all_samples} and \ref{table:beaver_sample_soil_Soverall_Cweather_controls}, I run a jackknife resampling exercise, in which I repeat the main specification in Table \ref{table:beaver_sample_soil_Soverall_Cweather_controls}, column (4), leaving one landscape grid cell out each time. The resulting distribution (Fig. \ref{fig:est-main-jackknife}) is spread evenly around the coefficient reported in Table \ref{table:beaver_sample_soil_Soverall_Cweather_controls}, with the tails extending approximately 0.1 p.p. in either direction. 

Because beavers may affect land use through multiple channels (including grazing, tree felling, flooding, and river bank collapse), I perform a provisional test for one commonly cited mechanism: flooding. Using the hydrometry measures from in-situ monitoring stations, I regress river level and flow rate on beaver treatment. In Table \ref{table:beaver_main_overall}, I report results using a small number of landscape grid cells which contain monitoring stations. The imprecise estimates indicate a potential small decrease in both river level and flow rate. While the qualitative effect is consistent with evidence that beaver dams lower water levels and peak flow rates downstream \citep{swinnen_environmental_2019}, the distribution of monitors relative to beaver dams is unknown, rendering the interpretation unclear. Still, if one assumes \textit{any} downstream-of-dam monitors in the data, then the effect may be a lower bound. But the effect does not provide clear evidence of flooding, which would most likely appear as an increase in the maximum observed river levels \textit{upstream} of the dam. 

