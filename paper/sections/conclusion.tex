A large literature has established that natural capital enters the economic production function. But there remain questions on the effect of wildlife on productivity. While some provide beneficial ecosystem services, others disrupt economic operations (e.g., predators feeding on livestock). Because agents do not typically know the true direction on this effect \textit{ex-ante}, many risk-averse operators will undertake potentially welfare-reducing control operations to neutralize potential threats. Using the recent case of Scottish beaver reemergence, after a local extinction centuries earlier, I provide evidence for the effect of one keystone species and ecosystem engineer on agricultural land use. On the extensive margin, land devoted to agriculture increases significantly in response to beaver entry, relative to comparison landscape patches that did not experience beaver habitation. Consistent with literature on beaver ecology and beaver-farm interactions, the positive effect is driven by landscape patches directly adjacent to watercourses with high arable cropping suitability. Using hydrometry measures, I observe suggestive evidence that beavers do indeed alter their physical environment, lowering flow rates and water levels, though the small sample and unknown spatial distribution of monitoring stations relative to beaver dams limits interpretability. Ongoing analysis aims to provide further evidence of the mechanism causing the positive reduced-form effect on agriculture land utilization, includes tests of spatial reallocation of agricultural land and measurement error. 