I estimate response functions of both land used for agriculture and local environmental characteristics (namely, river level and flow intensity) to beaver colonization.

I construct a two-period grid cell panel, using a 1km$^2$ tessellation of the study area. Year $t \in [1990, 2000] \mapsto \gamma = 0$. For the sample including all ever-treated cohorts, $t \in [2020, 2022] \mapsto \gamma = 1$. For the sample including only 2012 and 2017 treated cohorts, $t \in [2017, 2022] \mapsto \gamma = 1$. For the sample including only 2012-treated cohort, $t \in [2012, 2022] \mapsto \gamma = 1$. To rule out one potential violation of model assumptions, in Fig. \ref{fig:outcome-pretrends}, I calculate trends in agriculture land share in the pre-treatment years for which I have land use data (1990 and 2000). All the treated cohorts display similar, slightly negative trends over this time, and while the control group is much less agriculture-intensive, its trend does not appear to significantly differ from those of the treated cohorts. In estimation, I test omitting cohorts and find similar results.

Because beaver habitation is typically generations long, and dams can persist long after resident beavers abandon sites, I treat beaver arrival as an absorbing treatment.

I assign hydrometry monitoring stations to the grid cell in which they reside. Because the beaver survey data does not permit me to locate dams, I do not distinguish between upstream and downstream stations. Assuming a random distribution of underlying dam locations relative to monitoring station placement, the equal presence of downstream and upstream stations may cancel out any signal.

%\subsection{Agricultural Land Use}

To measure the impact of beaver habitation on agricultural land use, I estimate the classic two-period difference-in-difference model

\begin{equation} \label{eq:main_beaver_eq}
y_{it} = \alpha_i + \gamma_t + \beta^{b}D_{it} + \mathbf{X}_{it} + \epsilon_{it},
\end{equation}
where $y_{it}$ is the agricultural land use outcome, $\beta^b$ is the effect of being treated by beaver presence, $D_{it}$ captures beaver treatment, $\mathbf{X}_{it}$ is a vector of local precipitation and temperature controls, and $\epsilon$ is a random error term. $\alpha_i$ and $\gamma_t$ capture grid cell and pre-and post-period fixed effects, respectively. 