
In the following section, I briefly describe the agriculture industry in Scotland; beaver ecology, tendencies in territorial expansion, and behavior, including potential threats to agricultural productivity; and the Scotland's history of beaver extermination and recent reemergence. 

\subsection{Scottish Agriculture}

\begin{itemize}
    \item Overview of Scotland's climate and thus what kinds of agriculture are most suited to it $\rightarrow$
    \item Methods of agriculture in Scotland (how much is highland herding, how much is lowland cropping on arable land).
    \item History of how agricultural land was developed and protected by floodbanks (here I'm thinking about the quote from ``Eager'' p. 204 by Andrew Bauer: ``A lot of our land was bogs and marshes until four hundred or five hundred years ago, when floodbanks were built and it was reclaimed"). Get some other sources to back that up.
\end{itemize}

\subsection{Beavers}

\begin{itemize}
    \item Habitat: Where do they live? What kinds of environmental variables influence their habitation \parencite{swinnen_environmental_2019}?
    \item Ecosystem engineers: Dam, burrow, and lodge construction
    \begin{itemize}
        \item Ecosystem service benefits citations here.
    \end{itemize}
    \item Familial structures $\rightarrow$ dispersion patterns
    \item Behavior: potential costs to agricultural productivity
    \begin{itemize}
        \item Burrowing $\rightarrow$ collapsing fields, floodbanks
        \item Dam building $\rightarrow$ backing up rivers and flooding fields
        \item Crop grazing
        \item Timber felling
    \end{itemize}
\end{itemize}

\subsection{Beavers in Scotland}

\begin{itemize}
    \item Beaver reintroductions in eastern Europe, moving west
    \item Debate over introduction in the 1990s, followed by sanctuaries (Ramsey estate) and Knapdale controlled reintroduction (2009)
    \item Unauthorized emergence in Tayside in the 2000s.
\end{itemize}