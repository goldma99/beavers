
% ----------EXAMPLES OF CITATIONS---------------- 
% I got a fact from \cite{fairfax_using_2018} and \cite{alakoski_distribution_2021}. I got another fact from this paper \parencite{bouwes_ecosystem_2016}. I also want to mention this paper. \footcite{bartak_spatial_2013}
% -----------------------------------------------

\begin{itemize}
    \item In the face of climate change, habitat destruction, and biodiversity loss, there is a lot of focus on wildlife reintroductions.
    \item In many cases, reintroduced wildlife will interact with the human environment, producing a range of positive and negative externalities (spreading disease, predators that eat livestock, highway collisions, ecology-disrupting brumbies, etc.). 
    \begin{itemize}
        \item These conflicts with human environments is due, in part, to the history of human development going hand in hand with the destruction of wild habitats (e.g., draining swamps to establish agriculture). 
    \end{itemize}
    \item One salient example are the ongoing conflicts over the coexistence of human agriculture and the beaver, an ecosystem engineer which alters the course of rivers, establishes wetlands, fells timber, and grazes on vegetation. While the benefits of beaver presence have been well-documented, reliable cost estimates are rare. Farmers are a frequent opponent to beaver presence (cite examples from the beaver conflict literature). In Scotland, farmers have expressed dismay over a recent reemergence of beavers (here, give context for Kennedy Brexit quote).  
    \item The recent unplanned, unauthorized establishment of a now-massive beaver population in the agriculturally valuable Scottish region around the River Tay (hereafter, ``Tayside'') presents a valuable natural experiment to estimate the general impact of beaver habitation on agricultural viability. Exogenous to ag policy, climate change (compare to Arctic expansion \citep{tape_expanding_2022}), or wildlife management regime.
    \item To my knowledge, only one study has assessed the agriculture impacts of the Tayside beavers \citep{hamilton_tayside_2015}. Detail how this paper expands on and improves Scott's estimation procedure.
    \item Using data on beaver expansion, agricultural activity, and river levels, I estimate the impact of the Tayside beavers on a range of farm outcomes.
    \item I find \textcolor{red}{X} effect.
    \begin{itemize}
        \item Compare my estimates to \citep{hamilton_tayside_2015} and others.
    \end{itemize}
    \item I contribute to the literature on:
    \begin{itemize}
        \item Beaver costs and benefits
        \item More broadly, wildlife reintroductions and cohabitation. 
        \begin{itemize}
            \item (cite a few examples of small-scale estimates, like \citep{hamilton_tayside_2015}. Maybe mention this\footnote{Mention \href{https://www.exeter.ac.uk/media/universityofexeter/research/microsites/creww/riverottertrial/appendix1/Beavers_and_Agriculture.pdf}{this} as an example and note that it was in the context of a minuscule and more controlled beaver population. In addition, the estimates are only at a few sites})
        \end{itemize}
        \item Agricultural damage response functions \todo{Find literature. Ask Eyal and Amir.}
    \end{itemize}
    
\end{itemize}