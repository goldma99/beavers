To estimate the impact of beaver reintroduction on agricultural land use, I obtain data on (1) the Tayside beaver expansion from three comprehensive regional surveys conducted over a decade, (2) high-resolution satellite-derived land use classifications, (3) soil data on agricultural land compatibility, (4) satellite-derived elevation and slope data, and (5) river level from a network of in-situ hydrometry monitors across Scotland. To measure the impacts of beaver entry on agriculture, I employ the land use data, exploiting its spatiotemporal variation, spanning pre- and post-beaver entry periods, to test for land use change. To assess potential damage mechanisms, I use the hydrometry data. I use high-resolution data on soil type, elevation, and slope to control for beaver and agriculture suitability.

\subsection{Beaver Expansion}
\begin{itemize}
    \item Describe survey methods, with attention to causal inference concerns
    \item Describe how to translate survey observations to population estimates
\end{itemize}

\subsection{Land Cover Maps}
\begin{itemize}
    \item Describe source/methods of UKCEH land cover maps
    \item time periods covered
    \item methods and satellite used
    \item description of land class(es) I use to capture potential agriculture. Explain how ag land is classified
    \item Describe any limitations or which direction the bias could go
    \item Check if there were changes in methodology over the course of the sample period.
\end{itemize}

\subsection{River Levels}
\begin{itemize}
    \item Describe hydrometry network and history
    \item Describe measurements collected, with special attention to the interpretation of river level
\end{itemize}

\subsection{Landscape characteristics}

Briefly describe: 

\begin{itemize}
    \item Soil
    \item Elevation from ASTER (https://lpdaac.usgs.gov/products/astgtmv003/)
    \item Slope calculated as first derivative of elevation 
    \item Temperature, precip, and leaf area index
\end{itemize}

% \subsection{Remote Sensing of Floods}