
% ----------EXAMPLES OF CITATIONS---------------- 
% I got a fact from \cite{fairfax_using_2018} and \cite{alakoski_distribution_2021}. I got another fact from this paper \parencite{bouwes_ecosystem_2016}. I also want to mention this paper. \footcite{bartak_spatial_2013}
% -----------------------------------------------

Economic production faces environmental conditions that threaten efficiency. Technology can neutralize such threats (for instance, irrigation enabling productive agriculture in arid regions), but economically desirable solutions may harm social welfare. One salient example is the conflict between wildlife and agriculture. The success of farming operations has historically hinged on the ability of the producer or its government to control habitats, cull livestock predators and crop grazers, repel potential vectors of disease transmission, and eliminate crop-eating pests with mass insecticide use. Such measures, however, may produce both ecologically and socially undesirable outcomes. Case studies abound. In the past two decades, the US Department of Agriculture's Wildlife Services has killed 56 million wild animals, including some otherwise protected by the Endangered Species Act, to protect agricultural assets, largely livestock (\cite{torrella_inside_2024}). Pesticides used to protect crop yields negatively impact human and animal health \citep{larsen_agricultural_2017}, leading in part to the formation of the modern American environmental movement \citep{woodwell_broken_1984}. The Chinese ``Four Pests Campaign'' encouraged the mass killing of sparrows, believed to feed on grain reservoirs, and inadvertently eliminated sparrows' valuable pest-control services, playing a role in the subsequent mass famine (CITE). 

Balancing economic production and ecosystem protection is complicated by climate change, habitat destruction, and biodiversity loss \citep{cardinale_biodiversity_2012}. Indeed, misunderstanding the role keystone species play in supporting their ecosystems can distort wildlife management policy; recent evidence has challenged long-held beliefs about so-called ``nuisance'' species (e.g., \cite{raynor_wolves_2021}). There remains, however, scant empirical evidence on how resource use adapts to changes in the composition of natural capital.

The beaver (\textit{Castor canadensis} in North America\footnote{Kuhl, 1820} and \textit{Castor fiber} in Europe and Western Asia\footnote{Linnaeus, 1758}) occupies the dual roles of agricultural menace and crucial ecosystem engineer. Since the expansion of agriculture in Europe and America, farm operators have often killed beavers and destroyed their colonies to avoid the flooding, crop grazing, and timber felling. After being hunted to near-extinction until the 19th Century, the beaver has been reintroduced in many parts of North America and Europe. In recent years, a plethora of positive externalities produced by beavers has been documented, from wetland preservation \citep{hood_beaver_2008}, temperature regulation (\cite{dittbrenner_relocated_2022}), and carbon storage (\cite{wohl_landscape-scale_2013}, \cite{johnston_beaver_2014}) to wildfire resistance \citep{fairfax_smokey_2020} and species richness \citep{wright_ecosystem_2002}.

A recent unplanned, unsanctioned reemergence of beavers in Scotland, where they had not appeared in centuries, illustrates this conflict. Consistent with historical examples of farmer-led opposition to beaver habitation, agricultural groups have opposed beaver incursion into the agriculturally valuable region around the River Tay (hereafter, ``Tayside''), with the president of the National Farmers Union, Scotland (NFUS) warning that beavers pose a greater threat than Brexit to his constituents \citep{castle_beavers_2021}. In nearby Southern England, which has seen several controlled releases, residents have posted public banners vowing to oppose beaver invasion (\cite{itooksomephotos_say_2022}).  In Bavaria, where beavers reemerged in the 20th Century, farming organizations campaigned for their complete eradication (\cite{campbell-palmer_managing_2015}). Despite the staunch opposition, Scotland granted beavers protected status in 2019, restricting farmers' latitude to kill beavers. Indeed, anecdotal evidence suggests heterogeneous effects, with some farmers living in proximity to newly arrived beavers reporting no adverse effects, (\cite{campbell_rd_naturescot_2012}), while others complain of enormous monetary damages incurred by beaver activity (\cite{hamilton_tayside_2015}).

The case of the Scottish beaver can inform wildlife management policy. Exogenous to agricultural policy, climate change,\footnote{Not all beaver expansion has been uncorrelated with the identification-confounding effects of climate change. In recent decades, the warming Arctic tundra has proved fertile habitat for beavers \citep{tape_expanding_2022}, which are further altering the environment via their colonies' methane production \citep{clark_beaver_2023}.} or wildlife management regime shifts, the rapid spread of the beaver in Scotland  allows one to estimate the causal impact on agricultural land use.

To identify beaver movement over time, I employ a set of periodic comprehensive regional surveys, conducted between 2012 and 2020. To test whether beaver presence affects cropping behavior, I match high-resolution land use data to beaver arrival. Because multiple channels exist by which beavers may harm agricultural operations, I provide suggestive evidence for the most commonly cited mechanism: flooding. Using a network of hydrometry monitoring stations, I measure changes in river level and flow in response to beaver presence.

I find mixed evidence for beaver impacts. In my preferred specification, beaver arrival is associated with a 1.7 p.p. \textit{increase} in the share of landscape grid cells devoted to agricultural land use, a 13.6\% change relative to the average agricultural land share in my sample. This effect is stable across a range of cohort and grid cell subsamples. River levels, which beavers have been found to alter \citep{swinnen_environmental_2019}, respond noisily to beaver introduction. Level averages reduce modestly and insignificantly, while level maxima (i.e., the highest level measured in a given month) rises slightly, though not significantly. In preliminary analysis, it remains unclear whether the beaver-caused increase in share of land devoted to agriculture reflects true changes in cropping, or either a) the greening effect of beaver colonization or b) reallocation of cropping within farm properties. Further study, using satellite data to measure water saturation and flooding directly at a finer scale, as well as agricultural census data on farm productivity and income, may clarify the causal relationship.

This paper contributes an understanding of how natural capital enters the agricultural production function. A large literature focuses on the impacts of weather and climate on agricultural yield and economic growth (\cite{mendelsohn_impact_1994}, \cite{schlenker_impact_2006}, \cite{schlenker_nonlinear_2009}, \cite{hsiang_causal_2014}, \cite{taylor_environmental_2021}), while a small but growing field of studies treat wildlife as natural capital inputs (\cite{frank_economic_2024}, \cite{rucker_colony_2019}, \cite{champetier_bioeconomics_2015}, \cite{kawasaki_impact_2023}, \cite{devkota_assessing_2024}). This study also adds to findings on the valuation and accounting of natural capital more broadly (\cite{lewis_nature_2024}, \cite{raynor_wolves_2021}, \cite{fenichel_measuring_2016}, \cite{fenichel_natural_2014} \cite{kareiva_natural_2011}). Finally, it contributes evidence of the environmental effects of beaver habitation to a mature literature in ecology, whose findings I discuss further in section \ref{sec:background-beavers}. To my knowledge, I am the first to study, in a quasi-experimental setting, the impact of the beaver on the extensive margin of resource (i.e., land) use.

The paper is structured as follows Section. \ref{sec:background} desscribes the Scottish agricultural industry, beaver ecology, and the recent beaver recolonization of Scotland. Section \ref{sec:data} reviews the data. Section \ref{sec:methods} describes my empirical approach. Section \ref{sec:results} presents results on reduced-form beaver impacts on agriculture, as well as suggestive evidence on mechanisms. Section \ref{sec:discussion} discusses implications for policy. \ref{sec:conclusion} concludes and points toward future research.